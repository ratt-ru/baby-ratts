\section{Foreword}
This document is meant as a checklist/primer for newly arriving RATTs. This is meant to cover all aspects of studying at RATT, so older RATTs, please feel free to extend with your advice.
Admin matters
When in doubt, please contact Zizipo Lusizi <z.lusizi@ru.ac.za> and Verushca Kiewiets <v.kiewiets@ru.ac.za> for all matters related to admin.

\section{Bursaries, registration and visas}
Your letter of offer should be sufficient for registration at Rhodes. For visa matters, please contact the International students office (\ratturl{https://www.ru.ac.za/internationalisation}) and/or Zizipo.

Please be advised that your bursaries can often be slow to arrive (as late as March of a given year). We work with several sources of funding, and bureaucracy can be merciless. The University registrar will be understanding about this (as long as you have the bursary offer letter), your landlord perhaps less so. There is a Postgraduate Advance account you can draw on for rent and such until your bursary arrives. This requires a few days to arrange, so please contact Zizipo in advance if you need an … advance.


\section{Personae and policies}
See \ratturl{https://ratt.center/everyone/} for a list of people. If you’re not already included on this list, please send an email to Lexy (andatilexy@gmail.com) with your picture and your fields of interest/area of study.

\subsection{Weekly group meeting and journal club }
The weekly group meeting / journal club is every Thursday at 12:00 in room 42, with a remote connection to Cape Town via the CISCO unit. Everyone is expected to attend. A journal club speaker is designated on a rotating basis, however, if you come across an interesting paper that you would like to report on for the group, feel free to volunteer out of turn.
For the students, the purpose of you speaking at the journal club is twofold: 
\begin{enumerate}[(a)]
    \item you get to introduce your colleagues to interesting material, 
    \item you get to practise your presentation skills. 
\end{enumerate}

Therefore 
\begin{enumerate}[(a)]
    \item make your presentation interesting (not just “in this paper they did this and that and got this” -- tell us why it’s important and why should we care), and 
    \item prepare slides. It is not acceptable to just share a PDF of the paper and scroll over it as you talk, even if some of your senior colleagues have gotten away with it in the past.
\end{enumerate}

\subsection{Students-only Journal club}
We encourage (by which we mean, it is mandatory! --Oleg) students to also take part in the students-only journal club which takes place every Friday at 11 am in room 42. The aim of these meetings is similar to the regular journal club, to give students an opportunity to learn and improve their presentation skills in a less intimidating setting, without the senior staff looking on. During these meetings, students are selected to present on a rotating basis (but are very much encouraged to volunteer themselves). A student can present on any of the following topics:
Their research project, i.e. a simple progress report, and they can also share any difficulties they are facing with the project.

Any paper the student thinks might be interesting to themselves and/or the group.
Any software tool they use regularly that they think can be beneficial to others in the group.
This could also be used as an opportunity to practise for an upcoming conference. New students should email Lexy (andatilexy@gmail.com) in order to be added to the mailing list (RATT-Students-journal-club@googlegroups.com) that we use for these meetings.
Computing matters

\section{Mailing lists}
ratt-omnibus@googlegroups.com is the universal RATT mailing list for everybody (students, staff and even ex-RATTs).
ratt-busy-weeks@googlegroups.com is for busy week-related matters.
If you haven’t already been added to both mailing lists, drop Oleg <osmirnov@gmail.com> an e-mail.
Also sign up for the RATT-students mailing list, through this form: \ratturl{https://lists.ru.ac.za/mailman/listinfo/ratt-students} 
Slack
This is the general RATT Slack workspace: \ratturl{https://ratt-rarg.slack.com/}. Send Oleg an e-mail if you haven’t already got an invitation. There are channels for various topics, including private channels for projects.

\subsection{RATT cluster and github access}
% EM: The RATT github account can be found at \ratturl{https://github.com/ratt-ru}
% To access the RATT cluster, you will need an SSH key. Please generate a passphrase-protected SSH key, and send the public part (\code{id_rsa.pub}) to Oleg.
% For how to do this, see \ratturl{https://docs.github.com/en/github/authenticating-to-github/generating-a-new-ssh-key-and-adding-it-to-the-ssh-agent}
% Regenerating SSH keys when you get a new laptop is not necessary. Your key is your key, not your laptop’s key. You can take it with you, and you can use it on multiple machines. Your key consists of two parts, \code{.ssh/id_rsa} and \code{.ssh/id_rsa.pub}. Just copy both over to your new .ssh directory. 
You will also need a github account. Make one if you don’t have one, and send the github ID to Oleg. And learn to use git and github (online tutorial, anyone?)
EM: Please find a nice github tutorial by Brian from RATT here
An additional collection of detailed git tutorials can  be found here.

\subsection{RATT cluster technical support}
If you need to report a problem, or request a feature (e.g. a package install) on the cluster, open an issue here:  \ratturl{https://github.com/ratt-ru/systems/issues}. However, it should also be possible to install and run the desired package locally using either Homebrew on Linux or making an ubuntu singularity container on your computer, installing the required software then running on the server. Please refer to this link on how to install and run Homebrew locally, and this link on how to install and use singularity containers. If you need assistance please do not hesitate to reach out on the Whatsapp group here.

For minor questions about the cluster, use the \code{\#systems-issues} channel on the Slack. If in doubt which, rather open an issue. 

For all other tech support not related to the RATT cluster, e-mail Rhodes IT support on support@ru.ac.za. You can also get support from Andy <A.Youthed@ru.ac.za> at the department of physics, or the library IT help desk. 

\subsection{Laptops}
All postdocs and students have a one-off equipment grant to buy a laptop (+monitors, external mice, laptop bags and all that). The maximum total budget is set each year by SARAO HCD, and Zizipo can advise on the exact amount (not everybody is funded by SARAO, but the equipment budgets are identical as a matter of policy). 
Within the budget, choice of laptop brand, model and configuration is completely up to you. Consult the other students for advice. The Rhodes IT shop will have some standard models which you can obtain very quickly, anything non-standard will need to be ordered and can take two or three weeks depending on availability.

In terms of operating system, Ubuntu is highly recommended, unless you really can’t live without a Mac. If you want to run Windows, please reformat your hard drive immediately, then think again. If you want to experience both Windows and Ubuntu, please follow the instructions here on how to dual boot Windows 10 and Ubuntu 22.04. If you don’t have a Mac but long to experience the joys of having one, please follow the instructions here to make your Ubuntu look and feel like a Mac, or at the very least, make Ubuntu look pretty. 

The number 1 overwhelming reason to use Ubuntu is the excellent KERN suite maintained by your colleagues Gijs and Athanaseus. Every radio astronomy package you can think of (and some you can’t) can be installed from KERN with zero hassle. 

You can also think about running a “Virtual Machine” (VM) on your favourite laptop (e.g. Virtual Box on a Mac host,  running the latest Ubuntu).  It can be useful if you do not own the computer, you will be able to export and redeploy your working environment on your next machine. Be aware that running a Virtual Box inside a host will pump some of your computing resources. Remember, best developers and scientists code in Linux. (Or just try to compile something on Mac OS and you will see that bash is not so unfriendly after all.)



\section{Software}

\subsection{LaTeX }
During your studies, you will need to produce notes, reports and articles. The standard way to write scientific articles is. This might look strange and difficult at first glance but it will quickly become your friend to cherish during your career.
The most straightforward way to install a working LaTeX distribution is to go for:

\begin{itemize}
    \item Ubuntu: TeXLive
    \item Mac OS: MacTex
    \item Windows: remember we talked about formatting? go check MiKTeX or TeXLive
\end{itemize}
For online application addicts, you can use Overleaf with a free plan (limited compilation time and project size but ok for most papers + thesis)
After choosing your installation, go check this latex tutorial.

\subsection{Spelling and Grammar}
You are strongly encouraged to use the online or offline app from Grammarly. The free version is already very useful. RATT has a commercial subscription to Grammarly with extended features, but this is limited to 10 users at a time, so access is only granted to those busy working on a paper or their thesis -- talk to Oleg when
it’s that time for you.

\subsection{Translation}
Google Translate has very much improved since the past years, but a recent alternative exists such as DeepL that can integrate seamlessly your OS

\subsection{Python}
Ubuntu: comes standard.
Mac OS: do not spend a second and install Anaconda (Python 3)
Setting up and using Python virtual environments: 
Report to this guide or this other guide
If you want to easily change python versions, please refer to this guide 
For python tutorials, please refer to this google drive 
EM: For other python learning tools, you can refer to caktus.ai at the python writer section or ChatGPT and simply type what you want the python code to do. Please keep in mind these tools may not be 100\% accurate so you will have to run the code to check. 

\subsection{Python notebooks / Jupyter / Jupyterlab}
(some motivational text to do notebooks to keep track of research and use it as a -reproducible- lab book).
Manage bibliography
Any good software for this? (I use Mendeley that is multiplatform and browser integration + smartphone)
EM: I personally prefer Zotero as it can save your papers on google drive (15 GB storage) and rename them with author, year and title. Please refer to this link on how to setup. You can export these papers as a biblatex file. 
(Note on BibteX)

\subsection{Other CS questions}
The missing semester of your CS education (MIT): addressing, linux, bash, editor, version control, debugging, profiling, …
Science

\subsection{Access knowledge}
Arxiv: When you start, you usually need to do some bibliography to get acquainted with your research topic. The best and easiest way to find tons of materials is to use ArXiv and in particular the astro-ph subsection. It is a -big- repository of most published papers in the field. When a research team manages to get a journal paper accepted by an editor, the authors usually submit a version of their article on ArXiv (you may also do it someday). Beyond astronomy and astrophysics you can also find a lot of materials in mathematics, signal processing etc.
Good habit: try to get the habit of consulting it ~daily~ to watch new papers, you might select an article there for your next journal club. You can access through the website, or decide to subscribe to the daily/weekly mail, or find an app to consult it from your phone.
For all the rest, follow what your supervisor recommends you to read. Some papers are not freely accessible, you may ask your supervisor or somebody from the lab to see if there is an active subscription for the journal you want to consult. Some ebooks and review articles are also available. Do not stop trying to find the content of a paper because it is not available to you, ask a colleague.
A collection of useful learning materials on radio astronomy and interferometry are listed on the RATT website. These are mainly supplementary video tutorials.
Radio science
NRAO Essential Radio Astronomy: ERA
Website of Tobias Westmeier: \ratturl{https://www.atnf.csiro.au/people/Tobias.Westmeier/index.php}
Interferometry
NASSP course (JG: which repo is the latest?) EM: Place holder link
Statistics
JG: Anyone has a good primer on statistics in astronomy for ~Master / PhD candidate?
EM: Potentially useful Bayesian Statistics resources
KT: Modern Statistical Methods for Astronomy, Feigelson and Babu

Other Useful Resources
EM: \ratturl{https://linktr.ee/ratt.center}

\section{Miscellaneous}
\subsection{Posters}
Project posters e.g. those for the bursary conference, can be printed at Rhodes through the Printing Service Unit. Please contact Verushca or Zizipo when the need arises, they will guide you on how to get this done.

\subsubsection*{International Students}

\subsubsection*{Registration}
Registration is done at the student bureau in the Eden Grove building. This is also where you’ll get your student card, proof of registration and proof of accommodation if you stay in Res. Please carry your passport on registration day. Once registered, you’ll have access to ROSS where you can book meals, change passwords etc.

\subsubsection*{Phone Number}
There are various telephone networks here: Vodacom, Telkom, MTN, Cellc, Rain etc. Some are better than others in terms of price and/or coverage. For example, Vodacom is known for good cell coverage but very expensive data. Cellc has good rates but poor coverage in some areas etc. Depending on your preferences, there is a variety of choices. You may be asked for the following to obtain a sim card:
\begin{itemize}
    \item Proof of accommodation/address
    \item Your original passport
\end{itemize}
You can get the proof of accommodation from your Landlord or agent, or from Rhodes if you chose to stay in one of the postgraduate residences. It is also possible to get the proof of accommodation from the city hall located on High St, Grahamstown, Makhanda, 6139. Just ask anyone you bump into in city hall to show you where you can get one. 

\subsubsection*{Banking}
In order to receive your bursary, you require a South African bank account. The main ones are FNB, ABSA, CAPITEC and NEDBANK. Since FNB is used by Rhodes, money seems to reflect faster there. However, the choice of bank is up to you. The following will be required:

\begin{itemize}
    \item A copy of Passport with a valid VISA (Please also carry the original passport and VISA as they will need to verify)
    \item A copy of Proof of Accommodation/Address - as well as the original
    \item A copy of Proof of registration in the university (not sure of this one)
    \item A valid South African Phone number
\end{itemize}
Here you can find an article of customers rating different banks from best to worst as of 2022






\subsection{Accommodation}
There are two types of accommodation for students in Makhanda: University residences (res) and digs.

\subsubsection*{Res}
These are student housing units provided by the university. They are furnished (i.e have a bed, beddings, stove, vacuum cleaner, fridge, microwave, reading desk and lamp etc) and have wi-fi. You are at liberty to bring your own beddings, but you must buy your own kitchen utensils; they are not included. Here, rent includes water and electricity. There are three university residences for postgraduate students:

\begin{itemize}
    \item \textbf{Oakdene:} Within campus. Consists of two or three bedroom flats, each with a common bathroom, kitchen and lounge area (with tv and couches) shared amongst the flatmates.

    \item \textbf{Celeste house:} Slightly outside campus. Similar to Oakdene but the lounge is a common room (ie accessible to all members of Celeste)

    \item \textbf{Post Graduate Village (PGV):} Further away from campus. However, these are single bedroom units i.e., you have no flatmate.
\end{itemize}

All these residences are differently priced. Therefore, please refer to the fee booklet for their prices. There are also options of having meals in the university dining halls. Usually, Drostdy hall is the post grad dining hall. The menus are here. However, these have to be booked 48 hours in advance on ROSS which you will only have access to once registered.

Reses also have a shared laundry area consisting of washing machines, dryers and a few hanging lines. Laundry soap/pegs are not provided. 

For students new to Makhanda, it is advisable to stay in res at least for the first year, as it is easier and faster to get settled and know the way around. Here, there are hall wardens and sub wardens to welcome and assist new students, as well as showing them around if need be. It is also easier to know people and make new friends this way. Being away from home, having such a community may be a helpful addition 🙂.

\subsubsection*{Digs/Private Accommodation}
Digs are usually shared houses/flats away from campus owned by individuals/families and can typically consist of between one bedroom and six bedrooms. They can either be furnished/unfurnished. Depending on their location and state and the number of people sharing the house/flat, these can be more expensive or much cheaper than Res. Additionally, be mindful that rent may or may not include electricity, water, and internet. Deposits are usually required beforehand in digs. 

There are various facebook groups and pages where vacant digs are advertised e.g. looking for digsmates in grahamstown, Rhodes University - flatmates, flats, digsmates, digs, grahamstown accommodation etc. The best time to look for a digs on facebook is October to December. This is the time when most students’ lease are coming to an end and the landlord or agent starts looking for someone for the following year. Its around this time when you can get good value for money when it comes to accommodation. There are also agencies and websites that advertise flats for rent such e.g.

\begin{itemize}
    \item palm golding on 30 Somerset St, Grahamstown, Makhanda, 6139, 
    \item JUSTPROPERTY on Shop 8A, Peppergrove Mall, 22 African St, Grahamstown, Makhanda, 6139 
    \item ReMAX on 22 African St, Grahamstown, Makhanda, 6139
    \item Oaktreeproperties on 67 African St, Grahamstown, Makhanda, 6139
    \item Property24 e.t.c.
\end{itemize}
For your own sake, if you do choose to stay in digs, ensure that there is at least a supplementary water tank in case of water shortages, and put your safety first! Feel free to ask ratt members living in Makhanda for this. You can also find them on Whatsapp here. 
Side note: Towards the end of the year, students do a digs clearence by selling their belongings and its at this time when you can find good? Second hand items. Please check these links to find second hand items if need be. 
\begin{itemize}
    \item Rhodes Buy and Sell
    \item Grahamstown Makhanda Second Hand
    \item Grahamstown Makhanda - BUY \& SELL
    \item Grahamstown Marketplace
\end{itemize}


\subsection{Rhodes University Oppidan Bus}
Students staying outside res are referred to oppidans. There is usually a 13? passenger taxi that comes every hour from 6 pm until midnight to take students from Rhodes to where they stay around Grahamstown. As of 2022, the taxi comes to pick up students outside Steve Biko building. You can join the UCKAR whatsapp group or the UCKAR facebook page where you will find fellow Rhodes oppidans who would most likely have updated information regarding this mode of transport. However, beware that weekends transportation times are from 6pm - 8pm and the taxi does not usually operate during the Rhodes vacation times.  

Foreign Exchange / Pocket Money
It can be really difficult to change money in South Africa on a student VISA. As a rule of thumb, and if possible, please carry some Rands from home before you arrive. Depending on your spending habits, R1000 can get you groceries and essential things for a few days until your bursary/advance is sorted out. If you do run out of money before you’re sorted out, please shout. Do not sleep hungry! We’re here

It is normally also possible to withdraw Rands from your home bank accounts through local ATMs if you have VISA or MasterCard. Money could also be withdrawn from the supermarket (PnP/Checkers). It is said to be cheaper this way, practically free of charge. Just go to the till and say that you’d like “cashback”. You’ll “pay” via a PoS and then they will give you cash.
Groceries

You can buy groceries from the supermarket. In Grahamstown, we have:

\begin{itemize}
    \item Pick n Pay on 22 African St, Grahamstown, Makhanda, 6139 (closes 9PM)
    \item Checkers on The Carlton Centre, 109 High St, Grahamstown, Makhanda, 6139 (Closes 8PM)
    \item Spar on 35 African St, Grahamstown, Makhanda, 6140 (Closes by 11PM - needs to be verified)
    \item Shoprite on Cnr West and Beaufort Street Market Square Shoprite Building, Makhanda, 6140 (closes 6PM)
\end{itemize}

\subsection{Water}
Please note that tap water in Makhanda is mostly not potable. You may be required to buy drinking water from supermarkets such as Pick n Pay (Going for R5 a litre) or Spar (R6 a litre - need to confirm).  As of 2022, Grahamstown water infrastructure is old and prone to breakdowns. This translates into periods of water shortages and hence, the importance of having a watertank or borehole as backup in your digs or res. However, the municipality does announce when to expect the shortage on the Makana Local Municipality facebook page. 

\subsection{Electricity}
Electricity in digs is usually prepaid, which means you buy tokens that will last you for some time. If you get lucky, your digs will have electricity and water included in the rent. Rhodes res do not require you to pay for electricity or water separately.  
In South Africa, we experience loadshedding periods when we have planned power outages. This happens in different provinces at different times as the country tries to ration power. The rationing comes about when the demand for electricity exceeds supply. To get updated, you can either download the Loadshedding EskomSePush (ESP) app on android, or check updates from the Hi-Tec Security facebook page (They also give updates on unplanned outages - accidents, e.t.c), or, if you’d prefer having this information integrated to your google calendar, please follow these instructions. 
Please note that South African plugs and sockets are round as shown on this image. If you cannot find such a plug, it should be fairly easy to find adapters locally, that are compatible with your plug, from the local electronics stores or the Clicks Pharmacy Grahamstown   

The department of physics has electricity and wifi even when there is loadshedding so you can work from here.

\subsection{Eduroam wi-fi}
There is very good wi-fi coverage all over campus. After registration, you will be able to use eduroam. Please refer to this article on how to set it up. If you need internet to connect to eduroam, a classic catch 22, you can get a temporary hotspot setup at the Rhodes library, the IT department, Andy at the department of physics can also help, or just ask any one of us with an ethernet port to setup a hotspot for you.

\subsection{What to Carry}
Feel free to carry essential spices, food or utensils that you may miss from home. Basically anything legal that you can get through the airport within weight limits. There are supermarket(s) near campus where you can get the usual toiletries. 
Summers (~Oct~Feb) can get quite hot and winters (~Jun~Aug) quite cold, please carry appropriate clothing for these temperature changes 😎.
